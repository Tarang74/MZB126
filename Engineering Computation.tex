\documentclass{article}
\usepackage{template}

\usepackage{chngcntr} % Reset counter within sections
\usepackage{multicol}

\counterwithin*{equation}{section}
\counterwithin*{equation}{subsection}

\pagestyle{fancy}
\setlength\headheight{24pt}

\lhead{\className}
\rhead{\leftmark}
\cfoot{\thepage}

\newcommand{\uniTitle}{Queensland University of Technology}
\newcommand{\className}{Engineering Computation}
\newcommand{\classTime}{Semester 2, 2021}
\newcommand{\classInstructorName}{Dr Michael Dallaston}
\newcommand{\authorName}{Tarang Janawalkar}
\newcommand{\authorStudentNumber}{n11032201}
\newcommand{\classCode}{MZB126}

\usepackage[
    type={CC},
    modifier={by-nc-sa},
    version={4.0},
    imagewidth={5em},
]{doclicense}

\date{}

\begin{document}
\begin{titlepage}
    \vspace*{\fill}
	\begin{center}
        \LARGE
        \textbf{\className}
        \texorpdfstring{\\}{ }
        \uniTitle
        \texorpdfstring{\\}{ }
        \texorpdfstring{\vspace{0.3in}}{ }
        \normalsize\textit{\classInstructorName}
        \texorpdfstring{\\}{ }
        \classTime
    \end{center}
    \begin{center}
        \textbf{\authorName}
    \end{center}
    \vspace*{\fill}
    \doclicenseThis
    \thispagestyle{empty}
\end{titlepage}
\newpage

\tableofcontents
\newpage

\section{MATLAB Functions}
\lstset{language=Matlab}
\begin{table}[H]
    \centering
    \begin{tabular}{c | c}
        \toprule
        Function Syntax & Function Output \\
        \midrule
        \begin{lstlisting}
y = sin(x)
        \end{lstlisting} & Sine with $x$ in radians. \\
        \begin{lstlisting}
y = sind(x)
        \end{lstlisting} & Sine with $x$ in degrees. \\
        \begin{lstlisting}
y = asin(x)
        \end{lstlisting} & Arcsine with $y$ in radians. \\
        \begin{lstlisting}
y = exp(x)
        \end{lstlisting} & $\e^x$. \\
        \begin{lstlisting}
y = log(x)
        \end{lstlisting} & $\ln{\left( x \right)}$. \\
        \bottomrule
    \end{tabular}
    \caption{Common mathematical function in MATLAB.}
\end{table}
All the above functions are element-wise.
\begin{table}[H]
    \centering
    \begin{tabular}{c | c}
        \toprule
        Function Syntax & Function Output(s) \\
        \midrule
        \begin{lstlisting}
A = zeros(m, n)
        \end{lstlisting} & Creates an $m \times n$ matrix containing zeros. \\
        \begin{lstlisting}
A = ones(m, n)
        \end{lstlisting} & Creates an $m \times n$ matrix containing ones. \\
        \begin{lstlisting}
I = ones(m)
        \end{lstlisting} & Creates an $m \times m$ identity matrix. \\
        \begin{lstlisting} 
a = linspace(a, b, x)
        \end{lstlisting} & Creates an evenly spaced vector with bounds $\left[a, b\right]$. \\
        \begin{lstlisting}
y = length(A)
        \end{lstlisting} & The largest dimension of $A$. \\
        \begin{lstlisting}
[m, n] = size(A)
        \end{lstlisting} & The dimensions of $A$. \\
        \begin{lstlisting}
y = min(a)
        \end{lstlisting} & The minimum value in the vector $a$. \\
        \begin{lstlisting}
y = max(a)
        \end{lstlisting} & The maximum value in the vector $a$. \\
        \bottomrule
    \end{tabular}
    \caption{Matrices and Arrays in MATLAB.}
\end{table}
When manipulating matrices, \ast, \^, perform matrix operations, while prepending an operator with a dot (.) performs an element-wise operation.
\section{Operations in MATLAB}
\begin{multicols}{2}
    \subsection{Conditional Operations}
    \begin{lstlisting}
if expression
    statements
else if expression
    statements
else
    statements
end
    \end{lstlisting}
    Code inside an \lstinline{if} statement only executes if the expression is true. Note that only one branch will execute depending on which expression is true.
    \subsection{Iterative Operations}
    \begin{lstlisting}
while expression
    statements
end
    \end{lstlisting}
    Statements inside a \lstinline{while} loop \linebreak execute repeatedly until the expression is false.
    \begin{lstlisting}
for index = values
    statements
end
    \end{lstlisting}
    Statements inside a \lstinline{for} loop execute a specific number of times, based on the length of \lstinline{values}.
\end{multicols}
\section{Differential Equations}
\begin{definition}
    A differential equation is an equation that involves the derivatives of a function as well as the function itself.
    An ordinary differential equation (ODE) is a differential equation of a function with only one independent variable.
\end{definition}
\subsection{Electrical Systems}

\section{First-Order Ordinary Differential Equations}
\subsection{Separable ODE}
\begin{equation*}
    \dv{y}{t} = F(t, y)
\end{equation*}
\begin{enumerate}
    \item Rewrite the equation in the form: $f(y)\dd{y} = g(t)\dd{t}$.
    \item Integrate both sides: $\int f(y)\dd{y} = \int g(t)\dd{t}$.
    \item Rearrange for the explicit form of $y(t)$.
\end{enumerate}
\subsection{Linear ODE}
\begin{equation*}
    \dv{y}{t} + P(t)y = Q(t)
\end{equation*}
\begin{enumerate}
    \item Determine the integrating factor: $R(t)=\e^{\int P(t) \dd{t}}$.
    \item Solve:
    \begin{equation*}
        y(t)=\frac{1}{R(t)}\int R(t)Q(t)\dd{t} + C
    \end{equation*}
\end{enumerate}
\begin{proof}
    To solve a first-order linear differential equation, determine an integrating factor $R(t)$ such that
    \begin{equation*}
        R(t)P(t) = \dv{R}{t}
    \end{equation*}
    Multiplying the equation by $R(t)$ gives
    \begin{align*}
        R(t)\dv{y}{t} + R(t)P(t)y &= R(t)Q(t) \\
        R(t)\dv{y}{t} + \dv{R}{t}y &= R(t)Q(t) \\
        \dv{t}\bigl(R(t)y\bigr) &= R(t)Q(t) \\
        \int \dv{t}\bigl(R(t)y\bigr) \dd{t} &= \int R(t)Q(t) \dd{t} \\
        R(t)y &= \int R(t)Q(t) \dd{t} \\
        y(t) &= \frac{1}{R(t)}\int R(t)Q(t) \dd{t} + C
    \end{align*}
\end{proof}
\subsection{Solution using Linearisation}
A function can be linearised by using its 1st degree Taylor polynomial near $a$.
\begin{equation*}
    f(x) \approx f(a) + f'(a)(x-a) + \mathcal{O}(x^2)
\end{equation*}
This new polynomial can be substituted to form a linear ODE, which can be solved using an integrating factor.
\newpage
\section{Second-Order Ordinary Differential Equation}
\subsection{Constant Coefficient Linear ODE}
\begin{equation*}
    a \dv[2]{y}{t} + b \dv{y}{t} + c y = Q(t)
\end{equation*}
where $a,\:b,\:c\in\mathbb{C}$ are constants.
\subsection{Principle of Superposition}
Given that $y_1(t)$ and $y_2(t)$ are both solutions to a linear homogeneous second-order ODE, any linear combination of these solutions is also a solution to the differential equation.
\begin{equation*}
    y_H(t) = y_1(t) + y_2(t)
\end{equation*}
\subsection{Homogeneous ODE}
\begin{definition}
    A homogeneous ODE has $Q(t)=0$, which gives
    \begin{equation*}
        a \dv[2]{y}{t} + b \dv{y}{t} + c y = 0
    \end{equation*} 
\end{definition}
This differential equation has a solution of the form:
\begin{equation*}
    y(t) = \e^{rt}
\end{equation*}

\end{document}